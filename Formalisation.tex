\section{Formalisation}
\label{sec:Formalisation}

Our formalisation defines a \emph{paradigm} as a set of 
\emph{characterising properties} $\mathsf{\Pi}$ that holds over a mathematical 
construction called a \emph{paradigmatic structure} $\mathsf{PS}\in \PS$. 
This structure describes a specific \emph{workflow} (or \emph{process}) that 
captures how specific \emph{formalisms instances} are combined together, 
through transformations, towards achieving a specific intent $\iota$ that is 
adequately characterised by the properties. When $\iota$ has a commonly agreed 
name (in a given context, or for a specific community), and $\mathsf{\Pi}$ holds 
on a specific $\mathsf{PS}$, then $\mathsf{PS}$ is said to \emph{qualify} as 
the paradigm $\iota$. The properties constituting $\mathsf{\Pi}$ may 
characterise the formalisms and/or the workflow defining a paradigmatic 
structure. 

Take as a simple example \emph{object orientation} as a fairly recognised 
intent, which may translate into the following list of properties: the 
existence of objects possessing an identity; a notion of class that types 
objects, and that defines data and functions of these objects; a relation on 
classes called inheritence that allows objects of one class to acquire 
properties of objects of another class; and the existence of a message passing 
mechanism between objects. 
\moussa{Maybe factor out this description in a previous section dedicated to 
provide the intuition about paradigms?}. 

