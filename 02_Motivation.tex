\section{Motivation}
\label{sec:Motivation}

To develop a \textsc{cps}, project managers and engineers need to select the most 
appropriate development languages, software lifecycles and ``interfaces'' to 
specify the different views, components and their interactions of the system 
with as little ``accidental complexity''  \cite{BrooksSilverBullet} as possible. 
For example, when it is known that system/software 
requirements are likely to change frequently during the project's course, 
selecting an Agile development process may help cope with evolution and 
change. If the system's behaviour requires that operations are triggered when 
data become available, similar to reactive systems, Data Flow 
languages may help specify the most critical parts of the software behaviour precisely, 
making it amenable for timing analysis. 

\textsc{mpm} requires to \emph{model everything explicitly}, using \emph{the 
most appropriate formalism(s)}, \emph{at the most appropriate abstraction level(s)} 
\cite{PhD:VanTandeloo:2017}. This suggests that a \emph{paradigm} is a 
placeholder for the properties in each of the dimensions described 
above: the \emph{formalisms}, the \emph{abstraction levels}, and the 
\emph{processes} used in the modelling activities. 
Ultimately, \textsc{mpm} aims to select, organise and manage the 
three dimensions above. We aim to clarify the formal foundations of 
\textsc{mpm} to help design supporting tools. In this paper, we ignore the abstraction dimension 
for now and focus on the two others, though we do motivate here why the three dimensions are necessary and  how they are related.

\noindent
\textbf{Formalisms/Languages.} The first dimension relates to what is  
known as \emph{Modelling Language Engineering}, i.e., the explicit modelling of 
the key components of (modelling) languages: concrete and abstract syntaxes, 
as well as semantics, together with the usual activities: 
analysis, simulation, execution, debugging, etc. that should be 
supported by tools. Such tools may be largely synthesized from high-level 
specifications of these languages (such as metamodels for abstract syntax) and their usage. \emph{All} modelling artefacts, including 
language specifications as well as their instances, are 
organised in a repository which we call the \emph{Modelverse}.
The state of this repository evolves over time as the artefacts in it are modified.

\noindent
\textbf{Processes.} Workflows or life-cycles are processes relating the various \textsc{mpm} activities.
In particular, the above three primitive activities are combined. This is often supported by
a toolchain whereby different tools support different activities.
% FTG+PM is not required here yet. 
%
% To capture this dimension, we rely on a structure called \textsc{ftg+pm} \cite{Mustafiz-etAl:2012}. 
% Modelverse may be queried to obtain 
% related to modelling languages, but also the processes themselves), providing 
% a snapshot at the current time of the relevant artefacts that may be helpful 
% (e.g. for the general tasks described above), or providing a partial projection 
%of the full Modelverse to be reasoned over. 
Processes may be \emph{descriptive}, charting the sequence of activities carried out
as well as the artefacts involved, \emph{proscriptive} by declaratively 
specifying constraints on the allowed activities and their combinations,
and \emph{prescriptive} allowing enactment.

\noindent
\textbf{Abstraction/Refinement, Architectural and View decomposition.} 
During the course of system development, three basic approaches are commonly combined to tackle 
complexity. One particular combination of these approaches leads to Contract-Based System Design  \cite{SystemDesignContracts}.

\noindent
\textbf{\emph{Model abstraction (and its dual, refinement)}} is used when focusing on 
    a particular set of \emph{properties} of interest. A relationship $A$ between a detailed model 
    $m_d$ and a more abstract model $m_a$ is an \emph{abstraction} with respect to a 
    \emph{set of properties} $\Pi$ if for all properties $\pi \in \Pi$, the satisfaction of $\pi$ by the more abstract $m_a$ implies the satisfaction of $\pi$ by the more detailed $m_d$. 
    % Note that the above refers to "Boolean properties". These are often constructed from
    % the more general "value properties" by checking whether the latter lie in a given interval. 
    % This requires an appropriate metric over the value properties' domain.
    % The advantage of using Boolean properties is that logical inference can be used, using 
    % the many implication relationships that exist between the (satisfaction of) the different properties.
    % Different types of logic may be used. When reasoning with an "open world" assumption 
    % for example, some Description Logic is typically used.
    This allows one to \emph{substitute} $m_d$ by $m_a$ whenever questions about the properties in $\Pi$ need to be answered. Substitution is useful as the analysis of properties on the more detailed model is usually more costly
    % Less feasible in the case the model is a real-world artefact and the experiment 
    % required to assess the properties is unethical or impractical.
    % Costly may mean the difference between feasible in a reasonable time when 
    % for example model checking, where there is a combinatorial state-space explosion.
    than on the abstracted model.
    % Note that the more detailed model does have some advantages. 
    % First, it will allow correct assessment of a larger set of properties. 
    % Correctness refers to the true semantics of the artefact in case of symbolic artefacts 
    % and to validity with respect to the real world in case of physical artefacts.
    % Second, and related to the first, the validity range of for example "first principle" models
    % is much wider. Mechanical engineers will for example start from very detailed finite element
    % models to get insight into whether an object can be treated as a rigid body.
    % Once this has been determined, design alternatives will be explored using a rigid body abstraction.
    Note that the abstraction relationship may hold between models in the same or in different formalisms, as long as for both, the semantics allows for the evaluation of the same properties.
    % Although whether two models are related through abstraction with respect to a set of properties
    % needs to, in principle, be checked through explicit computation of the properties (i.e., in the
    % semantic domain), 
    % formalisms are often designed such that the abstraction can be guaranteed if some syntactic 
    % properties (such as insideness in formalisms with hierarchy) are satisfied.
    
\noindent
\textbf{\emph{Architectural decomposition (and its dual, component composition)}} is used when
    the problem can be broken into parts, each with an appropriate \emph{interface}.
    Such an encapsulation reduces a problem to (1) a number of sub-problems, each requiring 
    the satisfaction of its own properties, and each leading to the 
    design of a component and (2) the design of an appropriate architecture connecting the components in such a way that the composition satisfies the original required properties.
    Such a breakdown often comes naturally at some levels of abstraction, using appropriate formalisms (which support hierarchy).
    This is for example thanks to locality or continuity in the problem/solution domain.
    % Note that the above describes a top-down workflow where decomposition of the requirements
    % leads to design of components followed by architectural composition of these components.
    % A bottom-up workflow is also possible, where existing components are combined to satisfy
    % full-system requirements.
    Note that the component models may be described in different formalisms, as long their interfaces match and the multi-formalism composition has a precise semantics.
    
\noindent
\textbf{\emph{View decomposition (and its dual, view merge)}} is used in the 
    collaboration between multiple stakeholders, each with different concerns.
    Each viewpoint allows the evaluation of a stakeholder-specific set of properties. 
    When concrete views are merged, the conjunction of all the views' properties must hold.
    In the software realm, IEEE Standard 1471 defines the relationships between viewpoints and 
    their realisations, views.
    Note that the views may be described in different formalisms.



% FTG+PM is not required here yet. 
% 
%This \textsc{ftg+pm} also enables reasoning over shared parts of, and 
%relationships between languages to give meaning to various models aimed at 
%being used together: for example, finding the highest language (in terms of a 
%mathematical order over the formalisms supported by these languages) to which 
%all relevant models may be mapped, enabling easier reasoning, analysis or 
%(co-)execution.

% Does the FTG+PM get introduced elsewhere though?

% Nice to add model transformations,but for lack of space, not mandatory.
%
%All three dimensions have at their core the use of \emph{model transformation}: 
%for achieving the various activities for engineering languages (cf. 
%\cite{J:Lucio-Amrani-etAl:2014}) and models in these languages, for specifying the 
%multi-view relationships between models, or for performing the various activities.
%Furthermore, they may also provide the basic manipulations necessary 
%for querying, extracting and projecting the \emph{Modelverse}.


\begin{table}[t]
  \caption{Properties of two paradigms: Object Orientation ($\mathsf{OO} 
  \cite{Wegner:1987}$) and Computer-Aided Design ($\mathsf{CAD} 
  \cite{B:Groover-Zimmers:2008}$)}
   \vspace{-1em}
   \begin{center}
     \begin{small}
      \begin{tabular}[t]{c l}
         \hline
         \multicolumn{2}{l}{$\iota_1$: Object Orientation ($\mathsf{OO}$)}\\
         \hline
         $\mbox{OO}_1$ & Possess the concepts of Object and Class\\
         $\mbox{OO}_2$ & Objects possess a state and a set of capabilities / 
operations \\
         $\mbox{OO}_3$ & Possess an inheritance mechanism\\
         $\mbox{OO}_4$ & Inheritance allows to reuse operations\\
         \hline\hline
         \multicolumn{2}{l}{$\iota_2$: Computer-Aided Design ($\mathsf{CAD}$)}\\
         \hline
         $\mbox{CAD}_1$ & Comprises concepts of (2D/3D) points and lines\\
         $\mbox{CAD}_2$ & Shapes are defined by lines\\
         $\mbox{CAD}_3$ & Supports transformation of shapes into (2D/3D) 
products\\
         \hline
      \end{tabular}
      \end{small}
   \end{center}
   \label{tab:Properties}
\end{table}
