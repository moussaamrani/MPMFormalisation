\section{Conclusion}
\label{sec:Conclusion}
We have presented a formalization of the notion of modeling paradigm and illustrated it with the simple CAD paradigm. It consists of a set of properties that can be evaluated on a modeling scenario to check if its formalisms and workflows exhibit the features characterizing the paradigm. Like for programming languages, formalizing modeling paradigms can help understanding modeling approaches and guide implementation of modeling scenarios of different contexts based on well- known paradigms.

Future work will consist of applying our definition to more complex modeling scenarios such as the ones for developing CPSs and further validate our definition and evaluate its usefulness. Furthermore, the composition of modeling paradigms can be studied to answers questions such as how to combine paradigms, and if such combinations are paradigms themselves.

\all{Merge discussion with conclusion? Remove this discussion and shift it to Future Work?}

\all{What about Related Work?}

\subsection{Is Multi-Paradigm a Paradigm in itself?}
\label{sec:Discussion-Multiparadigm}

\subsection{Multi-Formalism \emph{vs.} Multi-Paradigms}
\label{sec:Discussion-MFvsMP}

\subsection{Dynamic use of Workflows with viewpoints}
\label{sec:Discussion-Dynamic}
