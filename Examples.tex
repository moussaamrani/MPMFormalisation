\section{Running Example: Poor-Man CAD}
\label{sec:Examples}

Many engineering disciplines (civil, mechanical, electrical, among others) 
require an \emph{a priori} \emph{design}, i.e. a kind of specification, or plan, 
of the final object, be it a bridge, a car's clutcher, or a complex electronic 
card. From an \textsc{Mpm} viewpoint, designs play the role of \emph{models}: 
they abstract from the real product while still possessing interesting 
properties, making them cheaper to produce and easier to analyse. To support 
designers, Computer-Aided Design (\textsc{Cad}) promotes the use of computers to 
help create, modify, and analyse designs, but also optimise against known 
constraints from a given domain. \textsc{Cad} has significantly improved many 
engineering disciplines by improving the designs quality and increase the 
designers' productivity, and even becoming standard development tools for 
disciplines like electronics. \textsc{Cad} tools allow the \emph{visually} 
design in 2D, 3D and even 4D (adding the time dimension), based on various 
techniques such as Finite Element Methods, thus manipulating complex 
Differential Equations. 

For the purpose of this introductory work, we propose the Poor-Man \textsc{Cad} 
(abbreviated as \textsc{PCad}), a (toy) (multi-)formalism that allows designing 
2D geometric lines (noted $L$) representing arbitrary shapes, based on the 
definition of points (noted $P$) placed on a cartesian plan (defined by integer 
X- and Y-coordinates). When required, lines may be colored to emphasize 
surfaces of a shape or parts of a design. Colors (noted as $C$) are encoding as 
RGB integer triples for simplicity. The following Definition and Example 
provides a formal specification and a visual representation of a possible 
design.

\begin{Definition}[\label{def:PCAD}\textsc{Pcad} Designs]
   A \textsc{PCad} object (or design) $o\in \PCAD$ is a set of lines. A 
\emph{line} $l \in L$ is defined as a set of exactly two points. A \emph{point} 
$p\in P$ is a pair of integers. Similarly, a \emph{colored} \textsc{PCad} object 
$o\in \PCADC$ is a set of lines defined over \emph{colored} points.


The function $\SMapping$ maps an object $o\in \PCAD$ to the 
unions of points the lines constituting the objects are made of. A \emph{color} 
$c\in C$ is a triple of integers corresponding to its RGB encoding.
\begin{displaymath}
   \begin{small}
     \begin{array}{rcl rcl}
      \PCAD &\eqdef& \wp(L) & \PCADC &\eqdef& \wp(CL)\\
      L     &\eqdef& \{ \{p, p'\} \;|\; & CL &\eqdef& \{ (\{p, 
p'\}, c) \;|\;  \\
            &      & \quad p, p' \in P \} & & & p, p' \in P, c\in C\}\\
      P     &\eqdef& \mathbb{N} \times \mathbb{N} & C     &\eqdef& 
[[0..255]]^3
   \end{array}
   \end{small}
\end{displaymath}

The semantics of a \textsc{Pcad} object is captured by the function 
$\SMapping$, that maps each object to the points the lines constituting the 
objects are built upon. The semantics of a colored \textsc{Pcad} object is 
similar, except that lines defined with points of different colors are 
blackened.

\begin{displaymath}
   \begin{small}
     \begin{array}{rcl}
\SMapping & \colon & \!\!\!\begin{array}[t]{rcl}
         \PCAD &\to& L\\
         o &\mapsto& \bigcup o
      \end{array}\\\\
\SMapping & \colon & \!\!\!\begin{array}[t]{rcl}
         \PCADC &\to& (P \nrightarrow C)\\
         o &\mapsto& \left\{ \begin{array}{rclcl}
            (p, p') &\mapsto& c & \mathrm{if} & c = c'\\
            (p, p') &\mapsto& (0,0,0) & \mathrm{if} & c \neq c'\\ 
         \end{array}\right. \\
         & \multicolumn{2}{l}{\forall (\{p, p'\}, c), (\{p, p'\}, c') \in o}\\
      \end{array}
   \end{array}
   \end{small}
\end{displaymath}
\end{Definition}

\begin{Example}[PCAD]
   Figure \ref{fig:PCAD-Visual} presents a visual representation of a very 
simple \textsc{Pcad} object. 

\begin{figure}[t]
   \centering
%    \includegraphics[width=0.98\columnwidth]{}
   \caption{A simple PCAD Object named $\mathsf{Form}$.}%
   \label{fig:PCAD-Visual}%
\end{figure}
\end{Example}

As minimal activities on \textsc{Pcad} designs, one shall evidently 
\emph{create} designs, \emph{check} for ensuring they represent valid, viable 
physical objects, and ultimately \emph{manufacture} the physical objects from 
the designs.


% \subsection{Finite-State Automata}
% \label{sec:Examples-FSM}
% 
% Finite State Automata (\textsc{Fsa}), as originally defined by Moore 
% \cite{J:Moore:1956}, link states with transitions that carry a label. 
% \textsc{Fsa} describe discrete state-based computations. Many variations of 
% \textsc{Fsa} exist together with various semantics, among which the 
% ``word-accepting'' semantics is one of the most used \cite{}. 
% 
% The \UML State Machines \cite{TR:UML-2.5:2015} is the language used by many 
% tools for expressing, among other possibilities, the behaviour of a \UML 
% object. Beyond state-based computations, \UML State Machines add several new 
% concepts like hierarchical and orthogonal states, history, and the possibility 
% to interact with the environment through outputs.
% 
% \subsection{Java}
% \label{sec:Examples-Java}
% 
% Java \cite{B:Java:2019} is a modern object-oriented programming language that 
% has become widely used for a large variety of applications. 
% 
% Wegner criteria for OO \cite{Wegner:1987}
% 
% \subsection{The \textsc{Md}$\star$ Jungle of Acronyms}
% \label{sec:Examples-MD}
% 
% \cite{B:Brambilla-Cabot-Wimmer:2012}






\begin{table}[t]
   \begin{center}
      \begin{tabular}[t]{c l}
         \hline
         \multicolumn{2}{l}{$\iota_1$: State Automata ($\mathsf{SA}$)}\\
         \hline
         01 & Contains the concepts of State and Transition\\
         02 & Possess a Transition enabler\\
         \hline\hline
         \multicolumn{2}{l}{$\iota_2$: Object Orientation ($\mathsf{OO}$)}\\
         \hline
         01 & Possess the concepts of Object and Class\\
         02 & Objects possess a state and a set of capabilities / operations \\
         03 & Possess an inheritance mechanism\\
         04 & Inheritance allows to reuse operations\\
         \hline\hline
         \multicolumn{2}{l}{$\iota_3$: Computer-Aided Design ($\mathsf{CAD}$)}\\
         \hline
         01 & \\
         02 & \\
         03 & \\
         04 & \\
         \hline
      \end{tabular}
   \end{center}
   \label{tab:Properties}
   \caption{Properties of three paradigms: State Automata ($\mathsf{SA} 
\cite{J:Moore:1956}$); Object Orientation ($\mathsf{OO} \cite{Wegner:1987}$) 
and Computer-Aided Design ($\mathsf{CAD} \cite{}$)}
\end{table}
