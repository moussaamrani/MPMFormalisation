\section{Running Example: Poor-Man Computer-Aided Design}
\label{sec:Examples}

\moussa{\begin{enumerate}
   \item Explain the following concepts:
      \begin{itemize}
      \item Design
      \item Computer-Aided
      \item CAD Common Processes
      \end{itemize}
   \item Present PCAD informally: visual and mathematical syntax + 
mathematical semantics.
   \item Retrieve processes identified during CAMPaM and try to express them as 
FTG+PM.
\end{enumerate}}

% \subsection{Finite-State Automata}
% \label{sec:Examples-FSM}
% 
% Finite State Automata (\textsc{Fsa}), as originally defined by Moore 
% \cite{J:Moore:1956}, link states with transitions that carry a label. 
% \textsc{Fsa} describe discrete state-based computations. Many variations of 
% \textsc{Fsa} exist together with various semantics, among which the 
% ``word-accepting'' semantics is one of the most used \cite{}. 
% 
% The \UML State Machines \cite{TR:UML-2.5:2015} is the language used by many 
% tools for expressing, among other possibilities, the behaviour of a \UML 
% object. Beyond state-based computations, \UML State Machines add several new 
% concepts like hierarchical and orthogonal states, history, and the possibility 
% to interact with the environment through outputs.
% 
% \subsection{Java}
% \label{sec:Examples-Java}
% 
% Java \cite{B:Java:2019} is a modern object-oriented programming language that 
% has become widely used for a large variety of applications. 
% 
% Wegner criteria for OO \cite{Wegner:1987}
% 
% \subsection{The \textsc{Md}$\star$ Jungle of Acronyms}
% \label{sec:Examples-MD}
% 
% \cite{B:Brambilla-Cabot-Wimmer:2012}


% \subsection{Poor-Man Computer Aided Design (\textsc{Pcad)}}
% \label{sec:Examples-PCAD}

The \textsc{PCad} (the Poor-Man Computer-Aided Design) formalism is a toy 
example designed to capture simple two-dimensional drawings. The following 
definition formalises \textsc{PCad}'s structure.

\begin{Definition}[\textsc{Pcad} Formalism\label{def:PCAD}]
   A \textsc{PCad} object $o\in \PCAD$ is a set of lines. A \emph{line} $l 
\in L$ is defined as a set of exactly two points. A \emph{point} $p\in P$ is a 
pair of integers. The function $\SMapping$ maps an object $o\in \PCAD$ to the 
unions of points the lines constituting the objects are made of. A \emph{color} 
$c\in C$ is a triple of integers corresponding to its RGB encoding.
\begin{displaymath}
   \begin{array}{rcl}
      \PCAD &\eqdef& \wp(L)\\
      L     &\eqdef& \{ \{p, p'\} \;|\; p, p' \in P \}\\
      P     &\eqdef& \mathbb{N} \times \mathbb{N}\\
      C     &\eqdef& \{ (r, g, b) \;|\; r, g, b \in [0..255] \}\\
      \\
      \SMapping & \colon & \!\!\!\begin{array}[t]{rcl}
         \PCAD &\to& P\\
         o &\mapsto& \bigcup o
      \end{array}
   \end{array}
\end{displaymath}
\end{Definition}
Figure \ref{fig:PCAD} describes a metamodel for the \textsc{PCad} formalism in 
a \MOF-like language. 

\begin{figure}[t]
   \centering
   \includegraphics[width=0.98\columnwidth]{PCAD}
   \caption{A Simple Metamodel for \textsc{PCad} (cf. Definition 
\ref{def:PCAD}).}%
   \label{fig:PCAD}%
\end{figure}



\begin{table}[t]
   \begin{center}
      \begin{tabular}[t]{c l}
         \hline
         \multicolumn{2}{l}{$\iota_1$: State Automata ($\mathsf{SA}$)}\\
         \hline
         01 & Contains the concepts of State and Transition\\
         02 & Possess a Transition enabler\\
         \hline\hline
         \multicolumn{2}{l}{$\iota_2$: Object Orientation ($\mathsf{OO}$)}\\
         \hline
         01 & Possess the concepts of Object and Class\\
         02 & Objects possess a state and a set of capabilities / operations \\
         03 & Possess an inheritance mechanism\\
         04 & Inheritance allows to reuse operations\\
         \hline\hline
         \multicolumn{2}{l}{$\iota_3$: Computer-Aided Design ($\mathsf{CAD}$)}\\
         \hline
         01 & \\
         02 & \\
         03 & \\
         04 & \\
         \hline
      \end{tabular}
   \end{center}
   \label{tab:Properties}
   \caption{Properties of three paradigms: State Automata ($\mathsf{SA} 
\cite{J:Moore:1956}$); Object Orientation ($\mathsf{OO} \cite{Wegner:1987}$) 
and Computer-Aided Design ($\mathsf{CAD} \cite{}$)}
\end{table}
